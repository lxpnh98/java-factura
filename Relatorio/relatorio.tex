\documentclass[12pt]{report}

\usepackage[a4paper]{geometry}
\usepackage[utf8]{inputenc}
\usepackage[portuguese]{babel}

\newcommand\tab[1][0.5cm]{\hspace*{#1}}


\title{Projeto de Programação Orientada aos Objetos (MiEI) \\ 2017/2018}
\author{Alexandre Mendonça Pinho (a82441) \and Joel Filipe Esteves Gama (a82202) \and Tiago Martins Pinheiro (a82491)}
\date{\today}

\begin{document}
\maketitle

\tableofcontents

\chapter{Resumo do enunciado}
\label{sec:resumo}

\tab Neste projeto é-nos proposto o desenvolvimento de uma plataforma onde contribuintes possam gerir a informação das faturas que foram emitidas em seu nome. Os contribuintes podem ser do tipo individual ou do tipo empresa e cada um com especificações diferentes, quer a nível de características quer a nível de gestão das faturas.

\chapter{Arquitetura de classes utilizada}
\label{sec:arquitetura}
\tab Para o desenvolvimento deste projeto utilizamos seis classes principais: \textit{EstadoMenu, Plataforma, Contribuinte, AtividadeEconomica, Fatura e JavaFatura}.

A classe \textit{EstadoMenu} e as suas subclasses representam os diferentes estados do menu, que permite a interação do programa com o utilizador.

A classe \textit{Plataforma} representa o sistema em si, isto é, a base de dados onde está guardada toda a informação.

A classe \textit{Contribuinte} representa o contribuinte e guarda a informação associada a este. É uma classe abstrata pois não existe um contribuinte genérico real, ou é um indivíduo ou empresa ou administrador.

A classe \textit{AtividadeEconomica} e as suas subclasses representam atividades económicas. Cada atividade económica implementa um algoritmo de cálculo do valor da dedução fiscal. Existem cinco possíveis atividades económicas disponíveis na plataforma.

A classe \textit{Fatura} representa um objeto do tipo fatura.

Por fim, a classe \textit{JavaFatura} é onde está presente o ponto de entrada do programa.

\chapter{Descrição da aplicação}
\label{sec:descricao}
\tab Nesta plataforma temos várias funcionalidades disponíveis, funcionalidades essas que estão divididas em diferentes menus.

\section{Menu Principal - \textit{MainMenu}}
\tab No menu principal o utilizador tem cinco opções: \textit{fazer login, registar um novo contribuinte, carregar um estado, guardar um estado e sair}.

Escolhendo a opção \textit{Fazer login} e introduzindo em seguida um NIF e uma password, caso o login seja feito com sucesso, (isto é, o contribuinte já está registado no sistema) o utilizador é encaminhado ou para o menu de indivíduo ou para o menu de empresa, dependendo do tipo de contribuinte que fez login.

A opção \textit{Registar um novo contribuinte}, permite ao utilizador, assim como o nome indica, registar um contribuinte no sistema (indivíduo ou empresa) caso este ainda não esteja registado.

A opção \textit{Carregar estado} é usada para carregar para o sistema informações préviamente guardadas em ficheiro, informações essas que foram guardadas utilizando a opção \textit{Guardar estado}, que permite ao utilizador guardar o estado do sistema num ficheiro com um nome escolhido pelo mesmo.

Por fim, o utilizador pode sair escolhendo a opção \textit{Sair}.

\section{Menu de Administrador - \textit{AdministradorMenu}}
\tab No menu de administrador, apenas acessível pelo administrador do sistema, existem três opções: \textit{listar empresas com maior faturação, listar os dez contribuintes que mais gastam, e logout}.

A opção \textit{listar os dez contribuintes que mais gastam} permite ao administrador saber quais os dez contribuintes que mais gastam no sistema. Já a opção \textit{listar empresas com maior faturação} permite ao administrador ver as X (número introduzido pelo administrador) empresas que mais faturam e quais as deduções fiscais associadas a essas fatuas.

A opção \textit{logout} faz o administrador voltar ao menu principal.

\section{Menu de Indivíduo - \textit{IndividuoMenu}}
\tab No menu de indivíduo existem também três opções: \textit{verificar fatura, calcular valor de dedução total e logout}.

A opção \textit{Verificar fatura} abre o menu de fatura para uma fatura especificada pelo utilizador (através do Id da fatura).

A opção \textit{Calcular valor de dedução total} permite ao utilizador verificar qual a dedução que as faturas com o seu NIF representam.

A opção \textit{logout} faz o utilizador voltar ao menu principal.

\section{Menu de Empresa - \textit{EmpresaMenu}}
\tab O menu de empresa dispõe de várias opções: \textit{criar fatura, logout, listar faturas por valor, listar faturas por data, calcular total acumulado da empresa, listar faturas por contribuinte e valor e listar faturas por contribuinte e data}.

A opção \textit{criar fatura} permite à empresa criar uma fatura, associada a um contribuinte individual a partir do NIF (do indivíduo).

A opção \textit{logout} faz o utilizador voltar ao menu principal.

A opção \textit{listar por valor e listar por data} listam todas as faturas a que a empresa está associada, mas com a diferença que uma opção lista por ordem crescente de valor e a outra lista as faturas não só por ordem crescente de datas, mas também entre duas datas escolhidas pelo utilizador.

A opção \textit{calcular total acumulado da empresa} calcula o total faturado por uma empresa entre duas datas escolhidas pelo utilizador.

A opção \textit{listar faturas por contribuinte e valor e listar faturas por contribuinte e data} listam todas as faturas a que a empresa está associada, mas desta vez, primeiro as faturas são ordenadas por contribuinte e só depois ordenadas em ordem crescente de valor e ordem crescente de datas (também entre duas datas escolhidas pelo utilizador).
\section{Menu de Fatura - \textit{FaturaMenu}}
\tab No menu de fatura existem três opções: \textit{imprimir fatura, alterar atividade econónica e voltar}.

A opção \textit{imprimir fatura} permite ao utilizador imprimir a informação referente à fatura escolhida.

A opção \textit{alterar atividade económica} permite ao utilizador alterar a atividade económica associada à fatura escolhida.

A opção \textit{voltar} faz o utulizador voltar para o menu de indivíduo.

\chapter{Intrudução de novos tipos}
\label{sec:intro}
\tab A introdução de novos tipos de despesas na plataforma pode ser feita de forma simples adicionando uma subclasse da classe \textit{AtividadeEconomica}, implementando o seu método de calculo da dedução fiscal, e adicionando uma opção nos menus onde for necessário.

\chapter{Conclusão}
\label{sec:conclusao}

\tab Foi-nos proposto, como projeto de avaliação, criar uma plataforma onde contribuintes podem gerir a informação das faturas que foram emitidas em seu nome. A implementação foi feita na linguagem de programação Java, utilizando conceitos de programação por objetos e garantido a abstração e dados e o encapsulamento.

Para melhorar a funcionalidade do projeto, pode ser implementado um sistema mais dinâmico de criação de tipos de atividade económica, guardando os diferentes tipos de atividades económicas respetivos e algoritmos de cálculo da dedução num ficheiro de texto à parte. Assim, é possível adicionar e alterar tipos de despesa sem recompilar o programa.

\end{document}
